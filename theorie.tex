\subsection{Gaz parfaits}
Pour un gaz dit "parfait" présent en quantité n de moles, on peut définir son état à partir des 3 constantes suivantes:
\begin{itemize}
    \item{La pression p en Pascal}
    \item{Le volume V en mètre cube}
    \item{La température en degré Kelvin}
\end{itemize}

De manière plus formelle:
\begin{equation}
    p \cdot V = n\cdot R \cdot T [J]
\end{equation}

R est la constante des gaz: $R=8.314 [J\cdot K^{-1} \cdot mol^{-1}]$

Si l'une de ces grandeurs reste constante, les deux autres ne peuvent pas varier indépendamment l'une de l'autre. En fixant la température, la loi de Boyle-Mariotte est obtenue:
\begin{equation}
    \label{boylemariotte}
    p\cdot V = constante
\end{equation}

Cette loi peut être observée grâce à un thermomètre à gaz.
Le thermomètre à gaz est un instrument de mesure constitué d'un grand tube de verre scellé à l'une de ses extrémités. De l'autre côté, une goutte de mercure assure l'étanchéité. Ce mercure va donc se mouvoir en fonction du rapport entre la pression extérieure $p_0$ et intérieure et définir un volume intérieur $V_0$.\\
En pompant de l'air, il est possible de créer une dépression d'un côté du mercure et obtenir une pression: $p_0 \Delta p$.\\
Une pression est aussi exercée par l'air enfermé par le mercure:
\begin{equation}
    p_{Hg} = \rho_{Hg}\cdot g \cdot h_{Hg}
\end{equation}
Sachant que:
\begin{align*}
    &\rho_{Hg} = 13600 kg/m^3\\
    &g = 9.81 m/s^2\\
    &h_{Hg} = hauteur\ de\ la\ colonne\ de\ mercure
\end{align*}

Donc la pression dans la colonne d'air du côté scellé vaut:
\begin{equation}
    p = p_0 + p_{Hg} + \Delta p
\end{equation}

Pour trouver le volume associé, il suffit d'utiliser la hauteur et la section intérieure du tube de verre.
\begin{equation}
    V = \frac{\pi \cdot d^2 \cdot h}{4}
\end{equation}
Le diamètre intérieur du tube vaut: $d=0.27cm$.

Quelques rappels concernant les conversions d'unités de pression:
\begin{align*}
    &1Pa = 1 N/m^2\\
    &1bar = 10^5 Pa\\
    &1torr = 1mmHg = 133.3224 Pa\\
    &760mmHg = 760torr=1atm=101325Pa=1.01325bar
\end{align*}

\subsection{Régression linéaire}

Pour la regression linéaire, les formules suivantes peuvent être utilisées afin de calculer les incertitudes liées à la courbe générée.
Dans ces équations, n représente le nombre de mesures et p la pente.

\begin{equation}
    \label{regressionr}
    r=\frac{\sum_{k=1}^{n}(x_k-\bar{x})\cdot(y_k-\bar{y})}{\sqrt{\sum_{k=1}^{n}(x_k-\bar{x})^2\cdot\sum_{k=1}^{n}(y_k-\bar{y})^2}}
\end{equation}
\\

\begin{equation}
    \label{deltam}
    \Delta p=p\sqrt{\frac{\frac{1}{r^2}-1}{n-2}}
\end{equation}

\begin{equation}
    \label{deltah}
    \Delta h = \Delta p \cdot \sqrt{\frac{\sum_{k=1}^{n}x_k^2}{n}}
\end{equation}
